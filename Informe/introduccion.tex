\section{Introducción}

El presente informe detalla un proyecto realizado en el marco del cursado de la materia TIC y Geomática de la Facultad de Ingeniería de la Universidad Nacional de Entre Ríos, enfocado en el análisis de la distribución espacial de vectores del dengue utilizando tecnologías geoespaciales y análisis de imágenes satelitales.

El dengue, una enfermedad transmitida por mosquitos del género Aedes, representa un desafío significativo para la salud pública. El estudio de la distribución de estos vectores es esencial para la implementación de estrategias. La combinación de datos satelitales y herramientas geoespaciales permite explorar como diferentes factores afectan la proliferación de los mosquitos.

El objetivo principal de este proyecto es aplicar los conocimientos adquiridos en la materia en una problemática real. A partir de la búsqueda bibliográfica se plantea la replicación en un área de interés diferente (Oro Verde, Entre Ríos, Argentina) de un trabajo que caracteriza y modela la distribución espacial de los vectores del dengue.

Este informe detalla el proceso desde la obtención de imágenes satelitales hasta el procesamiento geoespacial de los datos, incluyendo la delimitación del área de estudio, la extracción y análisis de índices espectrales como el NDVI y NDWI, la clasificación de superficies utilizando técnicas avanzadas en el análisis de datos y finalmente la implementación de la simulación propiamente dicha.