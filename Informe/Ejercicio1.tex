\graphicspath{{Images/}}

\section{Preparación del proyecto QGIS}\label{cap:ej1}

% TODO: \usepackage{graphicx} required

\subsection{Preparación del proyecto QGIS}


En primer lugar cargué las diferentes capas vectoriales que proporcionaba el TP y haciendo uso de QuickMapServices añadí la imágen satelital, obteniendo como resultado la figura a continuación.

\begin{figure}[H]
	\centering
	\includegraphics[width=0.5\linewidth]{Images/Img1}
	\caption[]{Imágen satelital de Clorinda con las muestras positivas y de control.}
	\label{fig:img1}
\end{figure}

A continuación fui agregando las diferentes capas ambientales raster, cambiando la representación de las coordenadas y cambiando sus estilos a pseudocolores monobanda con colores que consideraba representativos de la información a representar.

\section{Desarrollo}

\subsection{Ejercicio 1}

Para conocer a que capa ambiental corresponde cada figura fui desactivando y activando siguiendo la lista presente en el enunciado.

\begin{itemize}
	\item 1 Distancia Agua $\rightarrow$ Figura \textbf{E}
\end{itemize}


\begin{figure}[H]
	\centering
	\includegraphics[width=0.5\linewidth]{Images/Img2}
	\caption{Capa distancia agua sobre Clorinda}
	\label{fig:img2}
\end{figure}

\begin{itemize}
	\item 5 Distancia Vegetacion Alta $\rightarrow$ Figura \textbf{D}
\end{itemize}

\begin{figure}[H]
	\centering
	\includegraphics[width=0.5\linewidth]{Images/Img3}
	\caption{Capa distancia vegetación sobre Clorinda}
	\label{fig:img3}
\end{figure}

\begin{itemize}
	\item 8 Temperatura Máxima Media $\rightarrow$ Figura \textbf{C}
\end{itemize}


% TODO: \usepackage{graphicx} required
\begin{figure}[H]
	\centering
	\includegraphics[width=0.5\linewidth]{Images/Img4}
	\caption{Capa temperatura máxima media sobre Clorinda}
	\label{fig:img4}
\end{figure}

\begin{itemize}
	\item 7 Agua superficial $\rightarrow$ Figura \textbf{B}
\end{itemize}


% TODO: \usepackage{graphicx} required
\begin{figure}[H]
	\centering
	\includegraphics[width=0.5\linewidth]{Images/Img5}
	\caption{Capa agua superficial sobre Clorinda}
	\label{fig:img5}
\end{figure}


\begin{itemize}
	\item 9 Humedad ambiental $\rightarrow$ Figura \textbf{A}
\end{itemize}

% TODO: \usepackage{graphicx} required
\begin{figure}[H]
	\centering
	\includegraphics[width=0.5\linewidth]{Images/Img6}
	\caption{Capa humedad ambiental sobre Clorinda}
	\label{fig:img6}
\end{figure}

\subsection{Ejercicio 2}
Segun lo indicado en la teoría la selección de hábitats por parte de los organismos (vectores o reservorios de agentes patógenos) respondería a la distribución de los recursos, la calidad de los mismos y a la biología de los organismos.

\subsection{Ejercicio 3}
Factores que controlan la distribución y dispersión de una especie pueden ser la disponibilidad de alimento, la calidad y densidad de parches de hábitats, la presencia de refugios, la densidad de depredades, la existencia de gradientes físicos y la conectividad entre parcehes. (\textbf{Respuesta j. Todas son correctas})

\subsection{Ejercicio 4}
La cantidad de inidividuos que un ambiente puede soportar podría depender de, entre otros, de la disponibilidad de alimentos, de la disponibilidad de sitios de refugio-nidificación, de la densidad de depredadores, de la cantidad de especies competidoras y del efecto directo e indirecto de las variables climáticas. (\textbf{Respuesta i. a), b), c), d) y e) son correctas})

\subsection{Ejercicio 5}
Podrían incidir sobre las expansiones, retracciones o desapariciones de las áreas de distribución de las especies reservorio o vectores los efectos directos e indirectos de las variables climática y ambientales y la calidad del hábitat. (\textbf{Respuesta i. a), b), c), d) y e) son correctas})

\subsection{Ejercicio 6}
Fui generando los mapas de distribución con las distintas capas ambientales y ninguna de las opciones es correcta. El modelo que más se aproxima es con la capa de porcentaje urbano.

\begin{figure}[H]
	\centering
	\includegraphics[width=0.5\linewidth]{Images/Ej6}
	\caption{Mapa de distribución con la capa de porcentaje urbano.}
	\label{fig:ej6}
\end{figure}

\subsection{Ejercicio 7 y 8}
Tras generar el modelo con todas las variables disponibles observo la salida a través del archivo HTML. En el apartado de analisis de contribución de variables obtengo la siguiente tabla

% TODO: \usepackage{graphicx} required
\begin{figure}[H]
	\centering
	\includegraphics[width=0.7\linewidth]{Images/Contribuciones}
	\caption{Contribuciones de cada variable al modelo.}
	\label{fig:contribuciones}
\end{figure}

Siguiendo el criterio de que debe ser mayor a 14\% podemos concluir que las únicas variables que deberían participar en el modelo son la densidad urbana y el porcentaje urbano. Es decir, ninguna de las opciones del enunciado deberían ser consideradas.

\subsection{Ejercicio 9}
En función de la salida del modelo, para tomar decisiones debería tener en cuenta las siguientes consideraciones:

\begin{itemize}
	\item Utilizaría sitios positivos de control para evaluar el modelo antes de tomar decisiones.
	\item Realizaría un nuevo modelo solo con las variables más representativas y lo confrontaría con aquel que presenta todas las variables.
	\item Realizaría un muestreo a campo para corroborar la idoneidad de las predicciones generadas por el modelo.
	\item Tomaría el modelo para generar una zonificación que indique el riesgo de contraer la enfermedad en la localidad.
\end{itemize}

\subsection{Ejercicio 10}
Para el desarrollo de este ejercicio se exponen diferentes observaciones básicas sobre la epidemiología panorámica y se solicita que se selecciones verdadero o falso según corresponda:

\begin{itemize}
	\item Las enfermedades tienden a estar delimitadas solo por las condiciones climáticas. \textbf{Falso.}
	\item Las enfermedades tienden a estar geográficamente delimitadas. \textbf{Verdadero.}
	\item La variación espacial está relacionada a la variación de las condiciones físicas que soportan el patógeno y sus huéspedes o vectores. \textbf{Verdadero.}
	\item La variación espacial está relacionada a la varaición de las condiciones biológicas que soportan al patógeno y sus huéspedes o vectores. \textbf{Verdadero.}
	\item La delimitación de las condiciones ambientales y climáticas sobre mapas permiten predecir el riesgo. \textbf{Verdadero.}
\end{itemize}