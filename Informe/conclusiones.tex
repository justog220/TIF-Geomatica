\section{Conclusiones}

Este informe concluye el trabajo desarrollado en el marco del cursado de la materia TIC y Geomática. El objetivo central fue aplicar un modelo epidemiológico para predecir el crecimiento y difusión del vector que transmite el dengue en la localidad de Oro Verde, mediante el uso de sensores remotos y técnicas geoespaciales.

Se logró una delimitación precisa del área de estudio en Oro Verde, utilizando datos georeferenciados de ovitrampas y el procesamiento de estos. La obtención de imágenes satelitales de Landsat 8 fue fundamental para el desarrollo del trabajo.

La aplicación de diversos índices espectrales fue crucial para evaluar diferentes características del terreno, proporcionando una visión integral del área de interés, identificando factores que afectan la distribución del vector.

Un aspecto notable fue la implementación de técnicas matemáticas para suplir el submuestreo experimental. Estas técnicas permitieron interpolar y modelar la densidad espacial de huevos de mosquitos, proporcionando una estimación para áreas donde no se contaba con datos experimentales.

Se implementaron gráficos que permiten una visualización clara y comprensible de las variables involucradas en la distribución espacial del vector del dengue.

En resumen, este proyecto ha demostrado la efectividad de combinar datos satelitales y herramientas geoespaciales para analizar y comprender la distribución espacial de los vectores. Se sientan las bases para futuras investigaciones y estrategias de control en regiones afectadas por esta enfermedad.

Este trabajo representa la culminación del proceso formativo en la materia TIC y Geomática, aplicando los conocimientos adquiridos a una problemática real y relevante para la salud pública. La experiencia obtenida a lo largo del desarrollo de este proyecto ha sido invaluables, proporcionando una comprensión profunda de las técnicas geoespaciales y su aplicación práctica en el campo de la epidemiología.